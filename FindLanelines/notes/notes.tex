\documentclass[11pt, a4paper]{article}
\usepackage[utf8]{inputenc}
\begin{document}
\title{Finding Lane Lines Project}
\author{Samuel Navarro \\ Udacity Self Driving Car ND}
\date{\today}
\maketitle
	
	\section{Description}%
	\label{sec:description}

	I wanted to make the code in both C++ and Python. I'm a beginner at programming with C++ and because OpenCV has a very nice API both in Python and C++ this was a great oportunity for me.


	\section{Reflection}%
	\label{sec:problems}
	
	One of the problems I encounter it was that there where frames when you find the the fited line is not on the lane. Because of that I new that i needed some kind of average. I don't know if a simply avg or something like EWMA.


	Another problem I saw was the fact that the frames could be from different sizes. At the beginning I hardcoded the variable y\_fin so it can be roughly at the middle of the image. But when I started to work on the challenge video, I notice that the frames where different and in a real life scenario they will vary between videos. So I divided the frame.


	\subsubsection{Improvements}%
	\label{ssub:improvements}
	
		
	Some of the improvements are:
	\begin{itemize}
		\item GPU implementation.
		\item Benchmark code between CPU, GPU and cpp vs Python implementation.
		\item Better way to implements EWMA.
	\end{itemize}
	
\end{document}
