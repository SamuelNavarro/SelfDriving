\documentclass[11pt, a4paper]{article}
\usepackage[utf8]{inputenc}
\begin{document}
\title{Finding Lane Lines Project}
\author{Samuel Navarro \\ Udacity Self Driving Car ND}
\date{\today}
\maketitle
	
	\section{Description}%
	\label{sec:description}

	I wanted to make the code in both C++ and Python. I'm a beginner at programming with C++ and because OpenCV has a very nice API both in Python and C++ this was a great oportunity for me.

	However, this made me to waste a lot of time building the pipeline in C++ and because I didn't wanted to be more delayed, you can see that the Python code is very basic (and pretty ugly actually). But I just wanted to keep up and planned to make the Advanced Lane Lines with better code both in C++ and Python.

	\section{Pipeline}%
	\label{sec:pipeline}
	The current pipeline is as follows:
	\begin{itemize}
		\item Get edges with Canny edge detection after we convert the image to gray scale.
		\item Mask the image on the desired region.
		\item Get both lines (right and left) of the image based on the slope. 
			\begin{itemize}
				\item The criteria was hardcoded and there is room for improvements here.
			\end{itemize}
		\item Get $x_1$, $y_1$, $x_2$ and $y_2$ from the right and the left line.
		\item Fit the right and left lines with from the points that we obtained previously with `cv2.fitLine`. 
		\item Compute the lines to be drawed based on the average of all the previous lines. 
	\end{itemize}

	\section{Reflection}%
	\label{sec:problems}
	
	One of the problems I encounter it was that there where frames when you find that the fited line is not on the lane. Because of that I knew that I needed some kind of average. I think that a simple average would do part of the job but that wouldn't be enough because you could not deal with fast changes in the lines in the future. For that reason I believe that drawing the lines based on Exponential Moving Average would be betters because you can give more weight to the recent observations. For now, I proceeded with a simple average of the fitted lines both in the left and the righ line.


	Another problem I saw was the fact that the frames could be from different sizes. At the beginning I hardcoded the variable y\_fin so it can be roughly at the middle of the image. But when I started to work on the challenge video, I notice that the frames where different and in a real life scenario they will vary between videos. So I divided the frame.


	\subsubsection{Improvements}%
	\label{ssub:improvements}
	
		
	Some of the improvements are:
	\begin{itemize}
		\item Better criteria to divide the lines between right line or left image. Right now is hardcoded.
		\item Get the lines with EWMA; not just a simple average.
		\item Because I'm always curious about the differences between run you code with GPU or CPU, a GPU implementation would be desirable.
		\item Benchmark code between CPU, GPU and C++ vs Python implementation.
	\end{itemize}
	
\end{document}
